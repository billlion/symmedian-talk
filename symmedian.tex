\documentclass{beamer}

\usepackage{default}
\usepackage{color}
\usepackage{graphicx}
\usepackage{animate}
\usepackage{hyperref}

\begin{document}
\title{Where am I? The symmedian point!}   
\author{Bill Lionheart} 

\frame{\titlepage}




\begin{frame} \frametitle{What if the GPS goes off?}
\begin{center}
If the satellite navigation system (GPS,GLONASS) fails how can ships find their position?
\end{center}
\pause
\begin{center}
Use the old fashioned method of {\color{red}Celestial Navigation.} \\
Measuring the position of the stars and planets and an accurate clock to find our position.
\end{center}

\end{frame}

\begin{frame} \frametitle{How to do it}
\begin{enumerate}
\item<1-> Identify the stars or planets you can use.
\item<2-> Measure the angle it makes to horizon $H_o$ and note time. 
\item<3-> Look up the point on the earth where that object is directly overhead (GP)
\item<4-> From an {\em assumed position} AP work out the angle it should make to the horizon and its compass direction (azimuth).
\item<5-> Difference between measured $H_o$ and calculated $H_c$ angle and the bearing gives you a position line (LOP). 
\end{enumerate}
\vspace{-0.6cm}
\includegraphics[scale=0.19,angle=-90,origin=c]{HoHc.pdf}
%\includegraphics[scale=0.5]{intercept1-300x300}

%{\tiny Diagram from blueplanetcruisingschool.com}

\end{frame}


\begin{frame}\frametitle{Sextant animation}

%\url{https://en.wikipedia.org/wiki/Sextant#/media/File:Using_sextant_swing.gif}

\only<1>{\includegraphics[width=\linewidth]{sextant-0}}

\only<2>{\includegraphics[width=\linewidth]{sextant-1}}
\only<3>{\includegraphics[width=\linewidth]{sextant-5}}
\only<4>{\includegraphics[width=\linewidth]{sextant-10}}
\only<5>{\includegraphics[width=\linewidth]{sextant-15}}
\only<6>{\includegraphics[width=\linewidth]{sextant-16}}
\only<7>{\includegraphics[width=\linewidth]{sextant-18}}
\only<8>{\includegraphics[width=\linewidth]{sextant-20}}
\only<9>{\includegraphics[width=\linewidth]{sextant-25}}
\only<10>{\includegraphics[width=\linewidth]{sextant-26}}

\only<10-> {\tiny Animation from Wikipedia/Sextant(Marine) Joaquim Alves Gaspar CC BY-SA 2.5}


%  \animategraphics[loop,controls,width=\linewidth]{12}{sextant-}{0}{27}

\end{frame}

\begin{frame}\frametitle{Three lines don't meet at a point!}

\includegraphics[width=\linewidth]{threelines.eps}

\end{frame}

\begin{frame}\frametitle{Solve three equations in two variables}
\only<1->{$$
\begin{array}{c}
 a_{11} x_1+a_{12} x_2 =b_1\\
 a_{21} x_1+a_{22} x_2 =b_2\\
 a_{31} x_1+a_{32} x_2 =b_3\\
\end{array}
$$}

\only<2->{$$
A=\left(
\begin{array}{cc}
 a_{11} & a_{12} \\
 a_{21} & a_{22} \\
 a_{31} & a_{32} \\
\end{array}
\right), 
 x=\left(
\begin{array}{c}
 x_1 \\
 x_2 \\
\end{array}
\right),
b= \left(
\begin{array}{c}
 b_1 \\
 b_2 \\
 b_3 \\
\end{array}
\right)
$$}
\only<3->{$$ Ax=b$$}


\end{frame}

\begin{frame}\frametitle{Least squares solution}
\only<1->{$$
\begin{array}{c}
 d_1=a_{11} x_1+a_{12} x_2 -b_1\\
 d_2=a_{21} x_1+a_{22} x_2 -b_2\\
 d_3=a_{31} x_1+a_{32} x_2 -b_3\\
\end{array}
$$}
\only<2->{$$d = Ax-b$$}

\only<3-> {Minimize $$d_1^2 +d_2^2+d_3^2 = ||d||^2$$}
\end{frame}

\begin{frame}\frametitle{Least squares solution}
\only<1->{ $||d||^2\ge 0 $  }
\only<2->{ Differentiate to find minimum point }
\only<3->{ $||d||^2$ is quadratic in $x_1$ and $x_2$ }
\only<4->{ Derivative gives linear equations for minimum }
\only<5->{$$ A^T A x = A^Tb,\quad A^T=\left(
\begin{array}{ccc}
 a_{11} & a_{21} & a_{31} \\
 a_{12} & a_{22} & a_{32} \\
\end{array}
\right)$$}
\only<6->{$$ x= \left( A^T A\right)^{-1}A^Tb$$}


\only<7->{$$ 
\left( \begin{array}{cc} a & b \\c &d \end{array} \right)^{-1} = \frac{1}{ad-bc} 
\left( \begin{array}{cc} d & -b \\- c &a \end{array} \right)$$}
\end{frame} 


\begin{frame}\frametitle{Where is the centre of a triangle?}
\only<1>{\includegraphics{centroid1}}
\only<2>{\includegraphics{centroid2}}
\only<3>{\includegraphics[scale=1.5]{centroid3}}
\only<4>{\includegraphics[scale=1.5]{centroid4}\\
The bisectors of the angles meet at the centre of the inscribed  circle - the {\em incentre}.}
\only<5>{\includegraphics[scale=1.5]{centroid5}\\
The medians are lines joining vertices to midpoints of the opposite sides. They meet at the {\em centroid}, the centre of gravity}
\only<6>{\includegraphics[scale=1.5]{centroid6}\\
These centres are typically different}
\only<7>{\includegraphics[scale=1.5]{centroid7}\\
Now reflect the median line {\color{blue}(blue)} in the bisector {\color{red}(red)}. The {\color{green} green} line is a {\em \color{green}symmedian} line
}
\only<8>{\includegraphics[scale=1.5]{centroid8}\\
\vspace{-3cm}
 The {\color{green} green} lines meet at the {\em \color{green}symmedian} point. This is the point that minimizes the sum of squared distances from the sides. Note it is closer to the shorter side.
}
\end{frame}

\begin{frame}
\frametitle{So how many `centres of a triangle' are there? }
\vspace{-0.5cm}
\begin{center}
\only<1->{
There is a list: Clark Kimberling's  {\em Encyclopedia of Triangle Centers}. }

\only<2->{
The Symmedian point is number 6 on the list }

\only<3->{
\includegraphics[scale=0.4]{KimberlingCenters_800} \\ {\tiny  From:Weisstein, Eric W. "Kimberling Cent er."  MathWorld--A Wolfram Web Resource. }}

\only<4->{There are currently 11809}
\end{center}

\end{frame}

\begin{frame}\frametitle{And GPS?}
To find your position (including height) using GPS you need the distance to four satellites. The intersection of four planes in three dimensional space.

I will leave it to you to see if you can extend what we have done today to that case!
\end{frame}


\begin{frame}\frametitle{Notes}
\small
\begin{itemize}
\item If you want a quick overview of Celestial Navigation I recommend Blue Planet Cruising School's ``Celestial Navigation the missing introduction" \url{http://www.blueplanetcruisingschool.com/celestial-navigation-the-missing-introduction/} Diagrams are better than mine!
\item If you are interested in the question I raised about GPS and the symmedian of a tetrahedron, []lease have a go yourself. When you have done that you might consult Jawad Sadek, Majid Bani-Yaghoub, and Noah H. Rhee, Forum Geometricorum, Isogonal Conjugates in a Tetrahedron,
Forum Geo,Volume 16 (2016) 43–50.
\end{itemize}
\end{frame}
\end{document}